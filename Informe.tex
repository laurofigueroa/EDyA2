\documentclass[a4paper,12pt]{article}
\usepackage[spanish]{babel} % Soporte en español.
%\usepackage[latin1]{inputenc} % Caracteres con acentos.
\usepackage[utf8]{inputenc}

\newcommand{\bec}[1]{\begin{center}#1\end{center}}


\begin{document}

\title{Trabajo Práctico 2}
\author{José Agretti, Lauro Figueroa, Franco Vitali}
\date{\today}
\maketitle

\newpage

\section{Implementación de secuencias como listas.}

\subsection{Análisis de filterS}

\begin{math} 
    filter :: (a \rightarrow Bool) \rightarrow [a] \rightarrow [a] 

    filter f [] = [] 

    filter f (x:xs) = if f x then x : (filter f xs) else filter f xs
\end{math}


\subsubsection{Trabajo de filterS}


\begin{math}
    W_{filterS} (0) = k_1

    W_{filterS} (n) = 1 + W_{filterS} (n-1) 
\end{math}


Proponemos que el $W_{filterS}(n)$ es \theta(n) 

\bec{Prueba de trabajo de filterS}

\begin{math}
    W(n) = 1 + W(n-1)

    W(n) - W(n-1) = 1 (b = 1, y pol constate 1)
\end{math}{}

    Luego el polinomio caracteristico es $(x - 1) \cdot (x - 1) = (x - 1)^2$ con raiz 
    uno de multiplicidad 2.

    Entoces la solucion es de la forma $W(n) = c_1 \cdot 1^n + c_2 \cdot n \cdot 1^n$

    El sistema dado por las codiciones iniciales es

    \begin{math}
        W(0) = k_1 = c_1

        W(1) = c_1 + c_2 = 1 + k1 \Rightarrow c_2 = 1
    \end{math}
    
    Resolviendo obtenemos $c_1 = k1$ y $c_2 = 1$ ($k_1 \neq 0$)

    Por lo tanto $W_{filterS} = k_1 + n$ \Rightarrow $W_{filterS} \in \theta(n)$


\subsubsection{Profundidad de filterS}


\end{document}
